\documentclass[]{article}
\usepackage{lmodern}
\usepackage{amssymb,amsmath}
\usepackage{ifxetex,ifluatex}
\usepackage{fixltx2e} % provides \textsubscript
\ifnum 0\ifxetex 1\fi\ifluatex 1\fi=0 % if pdftex
  \usepackage[T1]{fontenc}
  \usepackage[utf8]{inputenc}
\else % if luatex or xelatex
  \ifxetex
    \usepackage{mathspec}
  \else
    \usepackage{fontspec}
  \fi
  \defaultfontfeatures{Ligatures=TeX,Scale=MatchLowercase}
\fi
% use upquote if available, for straight quotes in verbatim environments
\IfFileExists{upquote.sty}{\usepackage{upquote}}{}
% use microtype if available
\IfFileExists{microtype.sty}{%
\usepackage{microtype}
\UseMicrotypeSet[protrusion]{basicmath} % disable protrusion for tt fonts
}{}
\usepackage[margin=1in]{geometry}
\usepackage{hyperref}
\hypersetup{unicode=true,
            pdftitle={HW3},
            pdfborder={0 0 0},
            breaklinks=true}
\urlstyle{same}  % don't use monospace font for urls
\usepackage{color}
\usepackage{fancyvrb}
\newcommand{\VerbBar}{|}
\newcommand{\VERB}{\Verb[commandchars=\\\{\}]}
\DefineVerbatimEnvironment{Highlighting}{Verbatim}{commandchars=\\\{\}}
% Add ',fontsize=\small' for more characters per line
\usepackage{framed}
\definecolor{shadecolor}{RGB}{248,248,248}
\newenvironment{Shaded}{\begin{snugshade}}{\end{snugshade}}
\newcommand{\KeywordTok}[1]{\textcolor[rgb]{0.13,0.29,0.53}{\textbf{#1}}}
\newcommand{\DataTypeTok}[1]{\textcolor[rgb]{0.13,0.29,0.53}{#1}}
\newcommand{\DecValTok}[1]{\textcolor[rgb]{0.00,0.00,0.81}{#1}}
\newcommand{\BaseNTok}[1]{\textcolor[rgb]{0.00,0.00,0.81}{#1}}
\newcommand{\FloatTok}[1]{\textcolor[rgb]{0.00,0.00,0.81}{#1}}
\newcommand{\ConstantTok}[1]{\textcolor[rgb]{0.00,0.00,0.00}{#1}}
\newcommand{\CharTok}[1]{\textcolor[rgb]{0.31,0.60,0.02}{#1}}
\newcommand{\SpecialCharTok}[1]{\textcolor[rgb]{0.00,0.00,0.00}{#1}}
\newcommand{\StringTok}[1]{\textcolor[rgb]{0.31,0.60,0.02}{#1}}
\newcommand{\VerbatimStringTok}[1]{\textcolor[rgb]{0.31,0.60,0.02}{#1}}
\newcommand{\SpecialStringTok}[1]{\textcolor[rgb]{0.31,0.60,0.02}{#1}}
\newcommand{\ImportTok}[1]{#1}
\newcommand{\CommentTok}[1]{\textcolor[rgb]{0.56,0.35,0.01}{\textit{#1}}}
\newcommand{\DocumentationTok}[1]{\textcolor[rgb]{0.56,0.35,0.01}{\textbf{\textit{#1}}}}
\newcommand{\AnnotationTok}[1]{\textcolor[rgb]{0.56,0.35,0.01}{\textbf{\textit{#1}}}}
\newcommand{\CommentVarTok}[1]{\textcolor[rgb]{0.56,0.35,0.01}{\textbf{\textit{#1}}}}
\newcommand{\OtherTok}[1]{\textcolor[rgb]{0.56,0.35,0.01}{#1}}
\newcommand{\FunctionTok}[1]{\textcolor[rgb]{0.00,0.00,0.00}{#1}}
\newcommand{\VariableTok}[1]{\textcolor[rgb]{0.00,0.00,0.00}{#1}}
\newcommand{\ControlFlowTok}[1]{\textcolor[rgb]{0.13,0.29,0.53}{\textbf{#1}}}
\newcommand{\OperatorTok}[1]{\textcolor[rgb]{0.81,0.36,0.00}{\textbf{#1}}}
\newcommand{\BuiltInTok}[1]{#1}
\newcommand{\ExtensionTok}[1]{#1}
\newcommand{\PreprocessorTok}[1]{\textcolor[rgb]{0.56,0.35,0.01}{\textit{#1}}}
\newcommand{\AttributeTok}[1]{\textcolor[rgb]{0.77,0.63,0.00}{#1}}
\newcommand{\RegionMarkerTok}[1]{#1}
\newcommand{\InformationTok}[1]{\textcolor[rgb]{0.56,0.35,0.01}{\textbf{\textit{#1}}}}
\newcommand{\WarningTok}[1]{\textcolor[rgb]{0.56,0.35,0.01}{\textbf{\textit{#1}}}}
\newcommand{\AlertTok}[1]{\textcolor[rgb]{0.94,0.16,0.16}{#1}}
\newcommand{\ErrorTok}[1]{\textcolor[rgb]{0.64,0.00,0.00}{\textbf{#1}}}
\newcommand{\NormalTok}[1]{#1}
\usepackage{graphicx,grffile}
\makeatletter
\def\maxwidth{\ifdim\Gin@nat@width>\linewidth\linewidth\else\Gin@nat@width\fi}
\def\maxheight{\ifdim\Gin@nat@height>\textheight\textheight\else\Gin@nat@height\fi}
\makeatother
% Scale images if necessary, so that they will not overflow the page
% margins by default, and it is still possible to overwrite the defaults
% using explicit options in \includegraphics[width, height, ...]{}
\setkeys{Gin}{width=\maxwidth,height=\maxheight,keepaspectratio}
\IfFileExists{parskip.sty}{%
\usepackage{parskip}
}{% else
\setlength{\parindent}{0pt}
\setlength{\parskip}{6pt plus 2pt minus 1pt}
}
\setlength{\emergencystretch}{3em}  % prevent overfull lines
\providecommand{\tightlist}{%
  \setlength{\itemsep}{0pt}\setlength{\parskip}{0pt}}
\setcounter{secnumdepth}{0}
% Redefines (sub)paragraphs to behave more like sections
\ifx\paragraph\undefined\else
\let\oldparagraph\paragraph
\renewcommand{\paragraph}[1]{\oldparagraph{#1}\mbox{}}
\fi
\ifx\subparagraph\undefined\else
\let\oldsubparagraph\subparagraph
\renewcommand{\subparagraph}[1]{\oldsubparagraph{#1}\mbox{}}
\fi

%%% Use protect on footnotes to avoid problems with footnotes in titles
\let\rmarkdownfootnote\footnote%
\def\footnote{\protect\rmarkdownfootnote}

%%% Change title format to be more compact
\usepackage{titling}

% Create subtitle command for use in maketitle
\newcommand{\subtitle}[1]{
  \posttitle{
    \begin{center}\large#1\end{center}
    }
}

\setlength{\droptitle}{-2em}
  \title{HW3}
  \pretitle{\vspace{\droptitle}\centering\huge}
  \posttitle{\par}
  \author{}
  \preauthor{}\postauthor{}
  \date{}
  \predate{}\postdate{}


\begin{document}
\maketitle

\section{7.1}\label{section}

\emph{Describe a situation or problem from your job, everyday life,
current events, etc., for which exponential smoothing would be
appropriate. What data would you need? Would you expect the value of
alpha (the first smoothing parameter) to be closer to 0 or 1, and why?}

In my current role I work on forecasting how many field engineers of
different skillsets we will need on a weekly basis over the next two
years. This is done using a model which is trained on several years of
historical timesheet data. The type of work that these engineers is
doing is extremely seasonal and in addition to the seasonality, the
historic data can be quite `jagged'. There is not a lot of noise in the
data, but slightly smoothing the training data improves the performance
of the model. I think that the jaggedness in the unsmoothed data is a
result of a lot of processes which we cannot observe so it can just be
thought of as random noise in this case. I use an exponential smoothing
algorithm to smooth the historical data before feeding it into the
model. I do not know what the `alpha' value is in this particular case,
but I imagine that it is closer to 1 than 0 (assuming the equation is of
the form in the lectures) since there is not a lot of noise.

\section{7.2}\label{section-1}

\emph{Using the 20 years of daily high temperature data for Atlanta
(July through October) from Question 6.2 (file temps.txt), build and use
an exponential smoothing model to help make a judgment of whether the
unofficial end of summer has gotten later over the 20 years. (Part of
the point of this assignment is for you to think about how you might use
exponential smoothing to answer this question. Feel free to combine it
with other models if you'd like to. There's certainly more than one
reasonable approach.)}

We begin with loading required packages and the dataset. Also define the
CUSUM function to use and a nice function for showing regression lines
on scatter plots.

\begin{Shaded}
\begin{Highlighting}[]
\KeywordTok{library}\NormalTok{(ggplot2)}
\KeywordTok{library}\NormalTok{(reshape)}

\NormalTok{temps <-}\StringTok{ }\KeywordTok{read.table}\NormalTok{(}\StringTok{"7.2tempsSummer2018.txt"}\NormalTok{, }\DataTypeTok{header=}\OtherTok{TRUE}\NormalTok{)}

\NormalTok{temps}\OperatorTok{$}\NormalTok{DAY <-}\StringTok{ }\KeywordTok{as.Date}\NormalTok{(temps}\OperatorTok{$}\NormalTok{DAY, }\StringTok{'%e-%b'}\NormalTok{)}

\NormalTok{cusum <-}\StringTok{ }\ControlFlowTok{function}\NormalTok{(data, P, Q)\{}
\NormalTok{  ans <-}\StringTok{ }\KeywordTok{data.frame}\NormalTok{(}\DataTypeTok{S=}\KeywordTok{double}\NormalTok{(), }\DataTypeTok{alarm=}\KeywordTok{integer}\NormalTok{())}
\NormalTok{  ans[}\KeywordTok{nrow}\NormalTok{(ans)}\OperatorTok{+}\DecValTok{1}\NormalTok{,] <-}\StringTok{ }\KeywordTok{c}\NormalTok{(}\DecValTok{0}\NormalTok{,}\DecValTok{0}\NormalTok{)}
\NormalTok{  mu <-}\StringTok{ }\KeywordTok{mean}\NormalTok{(data)}
\NormalTok{  std <-}\StringTok{ }\KeywordTok{sd}\NormalTok{(data)}
\NormalTok{  C <-}\StringTok{ }\NormalTok{P }\OperatorTok{*}\StringTok{ }\NormalTok{std}
\NormalTok{  thresh <-}\StringTok{ }\NormalTok{Q }\OperatorTok{*}\StringTok{ }\NormalTok{std}
  \ControlFlowTok{for}\NormalTok{ (i }\ControlFlowTok{in} \DecValTok{2}\OperatorTok{:}\KeywordTok{length}\NormalTok{(data))\{}
\NormalTok{    S <-}\StringTok{ }\KeywordTok{max}\NormalTok{(}\DecValTok{0}\NormalTok{, ans[[i}\OperatorTok{-}\DecValTok{1}\NormalTok{,}\DecValTok{1}\NormalTok{]] }\OperatorTok{+}\StringTok{ }\NormalTok{(mu }\OperatorTok{-}\StringTok{ }\NormalTok{data[i] }\OperatorTok{-}\StringTok{ }\NormalTok{C))}
\NormalTok{    alarm <-}\StringTok{ }\NormalTok{S }\OperatorTok{>}\StringTok{ }\NormalTok{thresh}
\NormalTok{    ans[}\KeywordTok{nrow}\NormalTok{(ans)}\OperatorTok{+}\DecValTok{1}\NormalTok{,] <-}\StringTok{ }\KeywordTok{c}\NormalTok{(S, alarm)}
\NormalTok{  \}}
\NormalTok{  ans}
\NormalTok{\}}

\NormalTok{lm_eqn <-}\StringTok{ }\ControlFlowTok{function}\NormalTok{(linmodel)\{}
\NormalTok{  m <-}\StringTok{ }\NormalTok{linmodel;}
\NormalTok{  eq <-}\StringTok{ }\KeywordTok{substitute}\NormalTok{(}\KeywordTok{italic}\NormalTok{(y) }\OperatorTok{==}\StringTok{ }\NormalTok{a }\OperatorTok{+}\StringTok{ }\NormalTok{b }\OperatorTok\StringTok{ }\KeywordTok{italic}\NormalTok{(x)}\OperatorTok{*}\StringTok{","}\OperatorTok{~}\ErrorTok{~}\KeywordTok{italic}\NormalTok{(r)}\OperatorTok{^}\DecValTok{2}\OperatorTok{~}\StringTok{"="}\OperatorTok{~}\NormalTok{r2, }
                   \KeywordTok{list}\NormalTok{(}\DataTypeTok{a =} \KeywordTok{format}\NormalTok{(}\KeywordTok{coef}\NormalTok{(m)[}\DecValTok{1}\NormalTok{], }\DataTypeTok{digits =} \DecValTok{2}\NormalTok{), }
                        \DataTypeTok{b =} \KeywordTok{format}\NormalTok{(}\KeywordTok{coef}\NormalTok{(m)[}\DecValTok{2}\NormalTok{], }\DataTypeTok{digits =} \DecValTok{2}\NormalTok{), }
                        \DataTypeTok{r2 =} \KeywordTok{format}\NormalTok{(}\KeywordTok{summary}\NormalTok{(m)}\OperatorTok{$}\NormalTok{r.squared, }\DataTypeTok{digits =} \DecValTok{3}\NormalTok{)))}
  \KeywordTok{as.character}\NormalTok{(}\KeywordTok{as.expression}\NormalTok{(eq));                 }
\NormalTok{\}}
\end{Highlighting}
\end{Shaded}

Analysis begins with a visual inspection of smoothing on temperatures
from 1996. This uses alpha and beta parameters.

\begin{Shaded}
\begin{Highlighting}[]
\NormalTok{datchunk <-}\StringTok{ }\NormalTok{temps[,}\DecValTok{2}\NormalTok{]}
\NormalTok{smooth <-}\StringTok{ }\KeywordTok{HoltWinters}\NormalTok{(datchunk, }\DataTypeTok{gamma=}\OtherTok{FALSE}\NormalTok{)}
\NormalTok{smootheddatchunk <-}\StringTok{ }\KeywordTok{c}\NormalTok{(datchunk[}\DecValTok{1}\NormalTok{],datchunk[}\DecValTok{2}\NormalTok{],smooth}\OperatorTok{$}\NormalTok{fitted[,}\StringTok{'xhat'}\NormalTok{])}

\NormalTok{comparedat <-}\StringTok{ }\KeywordTok{data.frame}\NormalTok{(}\DataTypeTok{DAY=}\NormalTok{temps[}\StringTok{'DAY'}\NormalTok{], }\DataTypeTok{actual=}\NormalTok{datchunk, }\DataTypeTok{smoothed=}\NormalTok{smootheddatchunk)}

\NormalTok{meltedcompare <-}\StringTok{ }\KeywordTok{melt}\NormalTok{(comparedat, }\DataTypeTok{id =} \StringTok{'DAY'}\NormalTok{)}
\KeywordTok{ggplot}\NormalTok{(meltedcompare, }\KeywordTok{aes}\NormalTok{(}\DataTypeTok{x =}\NormalTok{ DAY, }\DataTypeTok{y =}\NormalTok{ value, }\DataTypeTok{colour =}\NormalTok{ variable, }\DataTypeTok{group=}\NormalTok{variable)) }\OperatorTok{+}
\StringTok{  }\KeywordTok{geom_line}\NormalTok{() }\OperatorTok{+}
\StringTok{  }\KeywordTok{ggtitle}\NormalTok{(}\StringTok{"Plot of actual temperature and smoothed temperature"}\NormalTok{)}\OperatorTok{+}
\StringTok{  }\KeywordTok{ylab}\NormalTok{(}\StringTok{"Temperature"}\NormalTok{) }\OperatorTok{+}
\StringTok{  }\KeywordTok{xlab}\NormalTok{(}\StringTok{"Day"}\NormalTok{)}
\end{Highlighting}
\end{Shaded}

\includegraphics{HW3_files/figure-latex/unnamed-chunk-2-1.pdf}

Next we apply HoltWinters smoothing to the raw temperature data for each
year. Each year of this smoothed data is then fed into the CUSUM model
with the same parameters as in homework 2.The output of this process is
a value for each year indicating how many days passed until the CUSUM
model registered a change (trip date). The plots below show a histogram
of the trip dates and a timeseries plot of them.

\begin{Shaded}
\begin{Highlighting}[]
\NormalTok{P <-}\StringTok{ }\DecValTok{1}
\NormalTok{Q <-}\StringTok{ }\DecValTok{5}
\NormalTok{firsttriplist <-}\StringTok{ }\KeywordTok{c}\NormalTok{()}
\ControlFlowTok{for}\NormalTok{ (cnum }\ControlFlowTok{in} \DecValTok{1}\OperatorTok{:}\NormalTok{(}\KeywordTok{ncol}\NormalTok{(temps) }\OperatorTok{-}\StringTok{ }\DecValTok{1}\NormalTok{))\{}
\NormalTok{  tempdat <-}\StringTok{ }\NormalTok{temps[,cnum }\OperatorTok{+}\StringTok{ }\DecValTok{1}\NormalTok{]}
  \CommentTok{# datchunk <- temps[,2]}
\NormalTok{  smooth <-}\StringTok{ }\KeywordTok{HoltWinters}\NormalTok{(tempdat, }\DataTypeTok{gamma=}\OtherTok{FALSE}\NormalTok{)}
\NormalTok{  smootheddatchunk <-}\StringTok{ }\KeywordTok{c}\NormalTok{(tempdat[}\DecValTok{1}\NormalTok{],tempdat[}\DecValTok{2}\NormalTok{],smooth}\OperatorTok{$}\NormalTok{fitted[,}\StringTok{'xhat'}\NormalTok{])}
\NormalTok{  ans <-}\StringTok{ }\KeywordTok{cusum}\NormalTok{(smootheddatchunk, P, Q)}
\NormalTok{  firsttrip <-}\StringTok{ }\KeywordTok{which}\NormalTok{(ans}\OperatorTok{$}\NormalTok{alarm }\OperatorTok{==}\StringTok{ }\DecValTok{1}\NormalTok{)[}\DecValTok{1}\NormalTok{]}
\NormalTok{  firsttriplist <-}\StringTok{ }\KeywordTok{c}\NormalTok{(firsttriplist,firsttrip)}
\NormalTok{\}}
\end{Highlighting}
\end{Shaded}

\begin{verbatim}
## Warning in HoltWinters(tempdat, gamma = FALSE): optimization difficulties:
## ERROR: ABNORMAL_TERMINATION_IN_LNSRCH
\end{verbatim}

\begin{Shaded}
\begin{Highlighting}[]
\KeywordTok{hist}\NormalTok{(firsttriplist, }\DataTypeTok{xlim=}\KeywordTok{c}\NormalTok{(}\DecValTok{40}\NormalTok{,}\DecValTok{140}\NormalTok{), }\DataTypeTok{xlab=}\StringTok{"First CUSUM trip - # of Days after July 1"}\NormalTok{, }\DataTypeTok{main=}\KeywordTok{sprintf}\NormalTok{(}\StringTok{"Smoothed Input. P: %s, Q: %s, min: %s, max: %s, mean: %s}\CharTok{\textbackslash{}n}\StringTok{"}\NormalTok{, P, Q, }\KeywordTok{min}\NormalTok{(firsttriplist), }\KeywordTok{max}\NormalTok{(firsttriplist), }\KeywordTok{mean}\NormalTok{(firsttriplist)))}
\end{Highlighting}
\end{Shaded}

\includegraphics{HW3_files/figure-latex/unnamed-chunk-3-1.pdf}

\begin{Shaded}
\begin{Highlighting}[]
\NormalTok{firsttripdf <-}\StringTok{ }\KeywordTok{data.frame}\NormalTok{(}\DataTypeTok{x=}\KeywordTok{seq_along}\NormalTok{(firsttriplist), }\DataTypeTok{y=}\NormalTok{firsttriplist)}

\NormalTok{linmodel <-}\StringTok{ }\KeywordTok{lm}\NormalTok{(y}\OperatorTok{~}\NormalTok{x, firsttripdf)}
\KeywordTok{ggplot}\NormalTok{(firsttripdf, }\KeywordTok{aes}\NormalTok{(}\DataTypeTok{x=}\NormalTok{x, }\DataTypeTok{y=}\NormalTok{y)) }\OperatorTok{+}
\KeywordTok{geom_point}\NormalTok{() }\OperatorTok{+}
\KeywordTok{geom_smooth}\NormalTok{(}\DataTypeTok{method=}\StringTok{'lm'}\NormalTok{) }\OperatorTok{+}
\KeywordTok{geom_text}\NormalTok{(}\DataTypeTok{x =} \DecValTok{13}\NormalTok{, }\DataTypeTok{y =} \DecValTok{100}\NormalTok{, }\DataTypeTok{label =} \KeywordTok{lm_eqn}\NormalTok{(linmodel), }\DataTypeTok{parse =} \OtherTok{TRUE}\NormalTok{)}
\end{Highlighting}
\end{Shaded}

\includegraphics{HW3_files/figure-latex/unnamed-chunk-3-2.pdf}


\end{document}
